\documentclass{article}
\usepackage{xeCJK}
\usepackage{ctex}
\usepackage{graphicx}
\renewcommand{\contentsname}{目录}
\renewcommand{\abstractname}{摘要}
\author{杨铭 - 5130379022\\
	李晟 - 5130379017\\
	张云翔 - 5130379012}
\title{HCI 课程选题提交}
\begin{document}
\maketitle
\section{问题陈述}
如何设计一个有趣的rpg游戏。
\begin{itemize}
	\item 需要有什么样的背景故事?
	\item 游戏的战斗机制能有怎么样的创新?
	\item 什么样的画面风格比较适合?
	\item 如何设计游戏角色的成长系统?
\end{itemize}
\section{头脑风暴}
\subsection{四元素混搭类型}
\subsubsection{技术}
\begin{itemize}
	\item 移动平台
	\item PC
	\item 多人在线
\end{itemize}
\subsubsection{机制}
\begin{itemize}
	\item 立体的游戏场景
	\item 回合制与动作结合
	\item 游戏场景随机生成
	\item 游戏技能相互影响
	\item 技能随机掉落
\end{itemize}
\subsubsection{故事}
\begin{itemize}
	\item 地牢冒险
	\item 星球大战背景
	\item 无限轮回
	\item 奇幻世界冒险
\end{itemize}
\subsubsection{美感}
\begin{itemize}
	\item 卡通风格
	\item 阴森恐怖风格
	\item 抽象风格
\end{itemize}
\subsection{其他游戏元素考虑}
\begin{itemize}
	\item 空
\end{itemize}

\section{基础设计}
\subsection{故事}
你是一个困在地牢中的勇士,你所知的信息有限,你需要不断冒险打倒敌人找到逃出牢笼的路径。然而最终会发现整个世界都是一个牢笼,每个人都在其中经历着轮回。
\subsection{机制}
游戏的每一局都是无限的轮回,打通一层地牢后,玩家可以继承上一层的状态。当玩家失败,玩家重新开始后不会继承之前的状态,但是会根据玩家以往的成就,开局角色会有额外的加成。游戏中人物的视野是有限的,移动时是按照回合制的方式,每个回合玩家和怪物根据自己的速度可以移动有限的路径。当玩家和怪物位置重叠时进入战斗状态,战斗是即时动作战斗,但处于战斗状态时,外面的回合也会根据时间进行,战斗时玩家和敌人处于一个封闭空间内,外面的敌人如果移动到相应的点位可以进入空间内和玩家进行战斗,玩家如果要出去必须杀死所有敌人或者打破空间壁。在消灭敌人或者探索地图时,玩家会有几率获得消耗品,武器和技能书。游戏中的技能分为回合技能和即时战斗技能,同类型的技能可以相互影响,产生组合技的效果。在每一层中都会有一个头目需要玩家去打败,打败后即可进入下一层。每一层的地图,敌人和道具都是随机生成的。
\subsection{美感}
使用卡通渲染,角色造型尽量诡异。
\subsection{技术}
利用unity引擎的移动端单人游戏
\section{透镜分析}
\subsubsection{本质体验}
\subsubsection{惊讶}
\subsubsection{趣味}
\subsubsection{好奇心}
\subsubsection{内生价值}
\subsubsection{解题}
\subsubsection{四类元素}
\subsubsection{全息设计}

\section{风险评估}
\begin{table}[htbp]
\centering
\caption{风险清单}
\begin{tabular}{|c|c|c|}
	\hline
	编号 & 风险 & 风险等级
	\hline
	1 & 不确定玩家是否会喜欢回合制移动和即时战斗混合的游戏机制 & 很严重
	\hline
	2 & 回合制移动和即时战斗混合的游戏机制的实现 & 很严重
	\hline
	3 & 随机生成地图的算法设计风险 & 严重
	\hline
	4 & 美术素材制作进度风险 & 严重
	\hline 
	5 & 角色技能系统和武器系统的设计实现风险 & 很严重
	\hline
\end{tabular}
\end{table}
\subsection{调整方案}
\begin{table}[htbp]
	\centering
	\begin{tabular}{|c|c|}
		\hline
		编号 & 调整方案
		\hline
		1 & 
		\hline
		2 & 
		\hline
		3 & 
		\hline
		4 & 
		\hline 
		5 & 
		\hline
	\end{tabular}
\end{table}

\subsection{迭代过程}

\subsection{团队分工}
\end{document}
