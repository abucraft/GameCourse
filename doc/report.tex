\documentclass{article}
%\usepackage{xeCJK}
\usepackage{CJK}
\usepackage{ctex}
\usepackage{graphicx}
\renewcommand{\contentsname}{目录}
\renewcommand{\abstractname}{摘要}
\author{杨铭 - 5130379022\\
	李晟 - 5130379017\\
	张云翔 - 5130379012}
\title{HCI 课程选题提交}
\begin{document}
\maketitle
\section{问题陈述}
如何设计一个有趣的rpg游戏。
\begin{itemize}
	\item 需要有什么样的背景故事?
	\item 游戏的战斗机制能有怎么样的创新?
	\item 什么样的画面风格比较适合?
	\item 如何设计游戏角色的成长系统?
\end{itemize}
\section{头脑风暴}
\subsection{四元素混搭类型}
\subsubsection{技术}
\begin{itemize}
	\item 移动平台
	\item PC
	\item 多人在线
\end{itemize}
\subsubsection{机制}
\begin{itemize}
	\item 立体的游戏场景
	\item 回合制与动作结合
	\item 游戏场景随机生成
	\item 游戏技能相互影响
	\item 技能随机掉落
\end{itemize}
\subsubsection{故事}
\begin{itemize}
	\item 地牢冒险
	\item 星球大战背景
	\item 无限轮回
	\item 奇幻世界冒险
\end{itemize}
\subsubsection{美感}
\begin{itemize}
	\item 卡通风格
	\item 阴森恐怖风格
	\item 抽象风格
\end{itemize}
\subsection{其他游戏元素考虑}
\begin{itemize}
	\item 空
\end{itemize}

\section{基础设计}
\subsection{故事}
你是一个困在地牢中的勇士,你所知的信息有限,你需要不断冒险打倒敌人找到逃出牢笼的路径。然而最终会发现整个世界都是一个牢笼,每个人都在其中经历着轮回。
\subsection{机制}
游戏的每一局都是无限的轮回,打通一层地牢后,玩家可以继承上一层的状态。当玩家失败,玩家重新开始后不会继承之前的状态,但是会根据玩家以往的成就,开局角色会有额外的加成。游戏中人物的视野是有限的,移动时是按照回合制的方式,每个回合玩家和怪物根据自己的速度可以移动有限的路径。当玩家和怪物位置重叠时进入战斗状态,战斗是即时动作战斗,但处于战斗状态时,外面的回合也会根据时间进行,战斗时玩家和敌人处于一个封闭空间内,外面的敌人如果移动到相应的点位可以进入空间内和玩家进行战斗,玩家如果要出去必须杀死所有敌人或者打破空间壁。在消灭敌人或者探索地图时,玩家会有几率获得消耗品,武器和技能书。游戏中的技能分为回合技能和即时战斗技能,同类型的技能可以相互影响,产生组合技的效果。在每一层中都会有一个头目需要玩家去打败,打败后即可进入下一层。每一层的地图,敌人和道具都是随机生成的。
\subsection{美感}
使用卡通渲染,角色造型尽量诡异。
\subsection{技术}
利用unity引擎的移动端单人游戏
\section{透镜分析}
\subsubsection{本质体验}
设想的体验是玩家挑战各种各样的怪物并通过打败它们来获得人物能力的提升和成就感。这种体验本质从游戏模式来说是RPG,是一种PVE,玩家需要通过提升自身实力和掌握一定的技巧来打败不同级别的敌对NPC。
这种体验的好坏应该是跟难度设定直接相关的,应当尽量让玩家在成功和挫败中曲折前进,使他们的前进路线满足“心流曲线”。为了实现这种体验,我们要设计较为平衡的游戏难度和多样化的战斗。
\subsubsection{惊讶}
我们游戏的本质是RPG,但是我们设计了一些跟一般RPG不太一样的机制:玩家的能力继承方式,回合制移动+即时战斗模式和灵活的技能学习、搭配等。能够给玩家带来全新的体验。
\subsubsection{趣味}
我们许多新奇的设定能够给玩家带来乐趣,特别是“去技能树”的设定,增加了游戏的自由度和随机性,技能可以被当做被动技能给其他技能提供加成这一特性,也给玩家带来了极大的自由发挥的空间。只是摆弄技能就是一大乐趣;回合制移动+即时战斗也不同于
以往的RPG游戏,将策略与即时战斗的快速反应融合在一起,更有战斗时小地图和整个场景的大地图融合的概念,增加了游戏的可能性,让游戏更有挑战性也更有趣味。
\subsubsection{好奇心}
也许无尽模式会让你感到厌倦,但是我们通过随机生成每一关的地图为游戏提供了更多的可能性。没有固定的技能树也是一方面,我们的技能是通过随机掉落的技能书来学习的,技能之间的组合也能生成更加强大的技能,你永远不清楚自己会在什么阶段学习到什么样的技能,也不清楚下一个拾取的技能将给自己带来怎样的提升。每次挑战新的BOSS时也将面对一个全新陌生的强力敌人,这些BOSS有什么技能?如何才能较好地打败它们?这些都能满足玩家的好奇心。
\subsubsection{内生价值}
游戏内的物品装备等都将成为玩家挑战下一目标的强力能力。获得强力的物品或技能意味着你的个人能力面板数值将更加高,面对下一个BOSS时将更加轻松。通关时的评分也是玩家关心的数值,因为它将成为反映自己游戏实力的象征。
\subsubsection{解题}
我们的游戏绝不是简简单单的“无双割草”,不同的BOSS将需要不同的技巧来打败。这些技巧可能需要你失败几次或者很细心地观察思考后才能较好地掌握。
\subsubsection{四类元素}
由于专业限制,游戏的画面和剧情元素将不会特别突出,我们将游戏的设计侧重于机制和技术上。当然我们尽量会制作成一款适合低画质特征的游戏。
\subsubsection{全息设计}
机制,主要是RPG这种游戏模式将是游戏的骨架,机制上我们的一些创新将为骨架提供血肉,丰满整个游戏,提高可玩性。
\section{风险评估}
\begin{table}[htbp]
\centering
\caption{风险清单}
\begin{tabular}{|c|c|c|}
	\hline
	编号 & 风险 & 风险等级
	\hline
	1 & 不确定玩家是否会喜欢回合制移动和即时战斗混合的游戏机制 & 很严重
	\hline
	2 & 回合制移动和即时战斗混合的游戏机制的实现 & 很严重
	\hline
	3 & 随机生成地图的算法设计风险 & 严重
	\hline
	4 & 美术素材制作进度风险 & 严重
	\hline
	5 & 角色技能系统和武器系统的设计实现风险 & 很严重
	\hline
\end{tabular}
\end{table}
\subsection{调整方案}
\begin{table}[htbp]
	\centering
	\begin{tabular}{|c|c|}
		\hline
		编号 & 调整方案
		\hline
		1 &
		\hline
		2 &
		\hline
		3 &
		\hline
		4 &
		\hline
		5 &
		\hline
	\end{tabular}
\end{table}

\subsection{迭代过程}

\subsection{团队分工}
\end{document}
